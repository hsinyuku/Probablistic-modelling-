% This part before the actual text is called preambel. It contains all sorts of specifications and packages (similar to packages in R) 
\documentclass[ % Specifications of the documents
	paper=a4, % paper size for printing
	bibliography=totoc, % literature is numbered and part of the table of contents (TOC)
	headings=optiontoheadandtoc
	]{scrartcl}

\usepackage{textgreek}
\usepackage{graphicx} % in case we want to use figures
\graphicspath{{fig/}{../fig /}} % I have made a folder called fog (see on the left hand side) where we can save our figures. We can then use figures using the \includegraphics-command within the figure-environment (if it comes to that, I will explain it further)



\usepackage{subfiles} % this package allows for a modular document structure, meaning that every part will have its own document
	
\usepackage{mystyle} % all other packages will be loaded in the file mystyle.sty, so there is less junk in this document

\title{Title goes here}
\subtitle{Totally fancy subtitle goes here, it can be a bit longer if we want it to be}
\subject{Probabilistic Modelling - Summer Semester 2020}
\author{Daniele Francario\textsuperscript{a}, Hsin-Yu Ku\textsuperscript{b}, Chân Le\textsuperscript{c}, Lukas Schmid\textsuperscript{d}}
\publishers{Supervisor: Prof. Dr. Burkhardt Funk}
\date{\today}

\setcounter{tocdepth}{2} % only heading of the second level appear in the table of contents (section and subsection, no subsubsection)

\addbibresource{sexy_modelers.bib} % this adds the data with all the citations

% Here begins the actual text.
\begin{document}

\pagenumbering{gobble} % no pagenumbers on the first few pages

\maketitle % this prints the title with the elements as they have been defined in the preambel
% this is a costume title page with our names
\vspace*{3cm}
\begin{center}
    \textit{\textsuperscript{a} Matr. No. 3038610, daniele.francario@stud.leuphana.de \\ \textsuperscript{b} Matr. No. 3038591, hsinyu.jade.ku@gmail.com \\ \textsuperscript{c} Matr. No. 3038545, le.chan@leuphana.de \\ \textsuperscript{d} Matr. No. 3038594, lukas.schmid@stud.leuphana.de}
\end{center}{}

\newpage

\tableofcontents % this prints the toc

\newpage

% \section*{Glossary} % here comes the glossary with all the abbreviations we used in the text and explain here

\listoffigures

\listoftables

\newpage

\setcounter{page}{1} % begin counting from one from this page
\pagenumbering{arabic} % no pagenumbers on the first few pages

% Writing of the report begins here - so it can be easily copy-pasted in a different location

\section{Introduction}
\subfile{tex/01_introduction}

\section{Methodology}
\subfile{tex/02_Methodology}

\section{Results}
\subfile{tex/04_Results}

\section{Model Simulation}
\subfile{tex/03_Model Simulation}

\section{Discussion}
\subfile{tex/05_Discussion}

\section{Conclusion}
\subfile{tex/06_Conclusion}

\section{Appendix}
\subfile{tex/07_Appendix}

\section{Declaration}
\subfile{tex/08_Declaration}

% \section{Reflection on the Group Work}

% \subfile{tex/how_to_cite.tex}

\newpage

\printbibliography % This prints the whole list of references.

\newpage

\pagenumbering{roman} % you guessed right: this command converts page numbers to roman numbers

% if we need an appendix
% \subfile{tex/07_appendix.tex} % Appendix at the very end of the document

\end{document}

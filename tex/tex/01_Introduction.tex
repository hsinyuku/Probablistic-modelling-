\documentclass[../main.tex]{subfiles}

\begin{document}

%\footnote{written by Daniele Francario}

At the date of writing of this report, the Sars-CoVid 19 Pandemic is rightfully earning its place as one of the deadliest ever experienced by humankind, with more than 16 million infected and 660.000 deaths registered since its beginning in December 2019. Predicting its impact over time, particularly the number of cases and deaths at the country and regional level, has proved particularly challenging and yet critically important to ensure adequate preparation of a country's healthcare infrastructure, given the fact that the pandemic only appears to keep growing on a worldwide scale, with a new peak of approximately 282.800 new cases worldwide in one single day having been reached on the 23rd of July, 2020.
One of the most common approaches to the structuring of public health strategies revolves around the use of the Case Fatality Ratio (CFR) calculated by dividing the number of reported deaths divided by the number of reported cases at a specific time point( which are readily available in most cases) as a reference to establish the number of individuals that will need to be hospitalized throughout the outbreak. It has however been highlighted how the CFR may not be a reliable estimate when planning for such contingencies, due to its sensitivity to the biases connected with right-censoring and preferential ascertainment. The first is due to the delay between symptom onset and death that characterizes most diseases, focusing only the number of death at a certain time might be misleading because it does not also consider the number of deaths that will occur among already infected individuals (right-censoring). Furthermore(preferential ascertainment).
An approach to solve this issue has been recently proposed in the paper by Hauser, Counotte, Margossian, Konstantinoudis, Low, Althaus, and Riou, entitled  "Estimation of SARS-CoV-2 mortality during the early stages of an epidemic: a modelling study in Hubei, China and northern Italy". The authors, whose publication represents the foundation for this project, try to accomplish this goal through the application of a to reconstruct the parameters of an extended compartmental SEIR mathematical model, and then use it to is then used to estimate the symptomatic Case Fatality Ratio (sCFR) and Infection Fatality Ratio (IFR) thus allowing for the construction of a decision support system capable of relying on available past data on cases and deaths caused by CoVid to predict the evolution of the outbreak and thus the organization of the healthcare system.
% Jade note_1
The first objective of our work was to replicate, with the introduction only of improvements to the overall structure of the original code, the authors' work for some of the countries they considered.
Our work also extends the accomplishment the authors in two different directions. First, we tried to use the original model to predict the same rates for gender groups, to observe whether it can be used to predict different infection and mortality dynamics among these.
We then attempt to introduce an additional improvement to the original model by considering the rate of reduction in transmissivity after the introduction of control measures to be specific for each age group and not equal across them. 


\end{document}